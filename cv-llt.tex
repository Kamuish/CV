%%%%%%%%%%%%%%%
% This CV example/template is based on my own
% CV which I (lamely attempted) to clean up, so that
% it's less of an eyesore and easier for others to use.
%
% LianTze Lim (liantze@gmail.com)
% 23 Oct, 2022
%

\newif\ifen
\newif\ifpt
\entrue % English enabled
% \pttrue % Portuguese disabled -> update document class if enabled


\documentclass[a4paper,skipsamekey,11pt,english]{curve}

% Uncomment to enable Chinese; needs XeLaTeX
% \usepackage{ctex}

% Language control
\newcommand{\langen}[1]{%
  \ifen\selectlanguage{english}#1\fi}
\newcommand{\langpt}[1]{%
  \ifpt\selectlanguage{portuges}#1\fi}






% Default biblatex style used for the publication list is APA6. If you wish to use a different style or pass other options to biblatex you can change them here. 
\PassOptionsToPackage{style=ieee,sorting=ydnt,
uniquename=init,
defernumbers=true,
maxnames=2}{biblatex}

% Most commands and style definitions are in settings.sty.
\usepackage{settings}

% If you need to further customise your biblatex setup e.g. with \DeclareFieldFormat etc please add them here AFTER loading settings.sty. For example, to remove the default "[Online] Available:" prefix before URLs when using the IEEE style:
\DefineBibliographyStrings{english}{url={\textsc{url}}}

%% Only needed if you want a Publication List
\addbibresource{firstauthor.bib}
\DeclareSourcemap{
  \maps[datatype=bibtex]{
    \map[overwrite]{
      \perdatasource{firstauthor.bib}
      \step[fieldset=keywords, fieldvalue={,self}, append]
    }
  }
}

\addbibresource{CoAuthor.bib}
\DeclareSourcemap{
  \maps[datatype=bibtex]{
    \map[overwrite]{
      \perdatasource{CoAuthor.bib}
      \step[fieldset=keywords, fieldvalue={,coAuth}, append]
    }
  }
}

\addbibresource{posters.bib}
\DeclareSourcemap{
  \maps[datatype=bibtex]{
    \map[overwrite]{
      \perdatasource{posters.bib}
      \step[fieldset=keywords, fieldvalue={,posters}, append]
    }
  }
}

\addbibresource{talks.bib}

% Add dates
\addbibresource{dates.bib} % generated/updated on the fly 
\newwrite\mydates
\newcounter{datecnt}
\newcommand\dates[1]  
{%
  \stepcounter{datecnt}%
  \immediate\write\mydates{@misc{date\thedatecnt,date={#1}}}%
  \citedate{date\thedatecnt}%
} 

\AtBeginDocument{\immediate\openout\mydates=dates.bib}
\AtEndDocument{\immediate\closeout\mydates}

%% Specify your last name(s) and first name(s) (as given in the .bib) to automatically bold your own name in the publications list. 
%% One caveat: You need to write \bibnamedelima where there's a space in your name for this to work properly; or write \bibnamedelimi if you use initials in the .bib
% \mynames{Lim/Lian\bibnamedelima Tze}

%% You can specify multiple names like this, especially if you have changed your name or if you need to highlight multiple authors. See items 6–9 in the example "Journal Articles" output.
\mynames{A. M. Silva,
  Silva/André\bibnamedelimi M.,
  Silva/André\bibnamedelima M.,
  }
%% MAKE SURE THERE IS NO SPACE AFTER THE FINAL NAME IN YOUR \mynames LIST


% Change the fonts if you want
\ifxetexorluatex % If you're using XeLaTeX or LuaLaTeX
  \usepackage{fontspec} 
  %% You can use \setmainfont etc; I'm just using these font packages here because they provide OpenType fonts for use by XeLaTeX/LuaLaTeX anyway
  \usepackage[p,osf,swashQ]{cochineal}
  \usepackage[medium,bold]{cabin}
  \usepackage[varqu,varl,scale=0.9]{zi4}
\else % If you're using pdfLaTeX or latex
  \usepackage[T1]{fontenc}
  \usepackage[p,osf,swashQ]{cochineal}
  \usepackage{cabin}
  \usepackage[varqu,varl,scale=0.9]{zi4}
\fi

% Change the page margins if you want
% \geometry{left=1cm,right=1cm,top=1.5cm,bottom=1.5cm}

% Change the colours if you want
% \definecolor{SwishLineColour}{HTML}{00FFFF}
% \definecolor{MarkerColour}{HTML}{0000CC}

% Change the item prefix marker if you want
% \prefixmarker{$\diamond$}

%% Photo is only shown if "fullonly" is included
% \includecomment{fullonly}
\excludecomment{fullonly}
\usepackage{orcidlink}

%%%%%%%%%%%%%%%%%%%%%%%%%%%%%%%%%%%%%%


\leftheader{%
  {\LARGE\bfseries\sffamily André M. Silva, \orcidlink{0000-0003-4920-738X}}

  \makefield{\faEnvelope[regular]}{\href{Andre.Silva@astro.up.pt}{\texttt{Andre.Silva@astro.up.pt}}}
  \makefield{\faGithub{}}
  {\href{https://www.github.com/Kamuish}{\texttt{Kamuish}}}
  \makefield{\faGlobe}{\url{https://kamuish.github.io/content/}}


  %% Next line
  \makefield{\faMapPin}{Instituto de Astrofísica e Ciências do Espaço,
  Rua das Estrelas, 4150-762 Porto, Portugal}

  \makefield{}{Ciencia ID: 2416-D5A6-DFA5}
  
  % You can use a tabular here if you want to line up the fields.
}

\rightheader{~}
\begin{fullonly}
\photo[r]{photo}
\photoscale{0.13}
\end{fullonly}

\title{Curriculum Vitae}

\begin{document}
\makeheaders[c]

\makerubric{employment}
\makerubric{education}

% If you're not a researcher nor an academic, you probably don't have any publications; delete this line.
%% Sometimes when a section can't be nicely modelled with the \entry[]... mechanism; hack our own and use \input NOT \makerubric
\makerubric{talks}
\section{Research outputs}

\mysubsection{\langen{First-author Papers}\langpt{Autor principal}}

\vspace*{0.2cm}

\circled{2}  André M. Silva et al, 2022 - 'A Novel Framework for Semi-{{Bayesian}} Radial Velocities through Template Matching', Astronomy \& Astrophysics (\textcolor{MarkerColour!80!black}{\scriptsize\faLink} DOI: \href{http://dx.doi.org/10.1051/0004-6361/202142262}{10.1051/0004-6361/202142262})

\vspace*{0.3cm}

\circled{1}  André M. Silva et al, 2020 - 'Archi: Pipeline for Light Curve Extraction of {{CHEOPS}} Background Stars', Monthly Notices of the Royal Astronomical Society (\textcolor{MarkerColour!80!black}{\scriptsize\faLink} DOI: \href{http://dx.doi.org/10.1093/mnras/staa1443}{10.1093/mnras/staa1443})

\vspace*{0.3cm}

\vspace*{0.2cm}

\mysubsection{\langen{Co-authored Papers}\langpt{Co-autor}}

\vspace*{0.2cm}

\circled{12}  Su{\'a}rez Mascare{\~n}o et al, 2024 - '{{TESS}} and {{ESPRESSO}} Discover a Super-{{Earth}} and a Mini-{{Neptune}} Orbiting the {{K-dwarf TOI-238}}', Astronomy \& Astrophysics ( \textcolor{MarkerColour!80!black}{\scriptsize\faLink} DOI: \href{http://dx.doi.org/10.1051/0004-6361/202348958}{10.1051/0004-6361/202348958})

\vspace*{0.3cm}

\circled{11}  Palethorpe et al, 2024 - 'Confronting Compositional Confusion through the Characterisation of the Sub-{{Neptune}} Orbiting {{HD}} 77946', Monthly Notices of the Royal Astronomical Society ( \textcolor{MarkerColour!80!black}{\scriptsize\faLink} DOI: \href{http://dx.doi.org/10.1093/mnras/stae707}{10.1093/mnras/stae707})

\vspace*{0.3cm}

\circled{10}  Campante et al, 2024 - 'Expanding the Frontiers of Cool-Dwarf Asteroseismology with {{ESPRESSO}}: {{Detection}} of Solar-like Oscillations in the {{K5}} Dwarf {\emph{{$\epsilon$}}} {{Indi}}', Astronomy \& Astrophysics ( \textcolor{MarkerColour!80!black}{\scriptsize\faLink} DOI: \href{http://dx.doi.org/10.1051/0004-6361/202449197}{10.1051/0004-6361/202449197})

\vspace*{0.3cm}

\circled{9}  De Beurs et al, 2024 - 'Characterization of {{K2-167}} b and {{CALM}}, a New Stellar Activity Mitigation Method', Monthly Notices of the Royal Astronomical Society ( \textcolor{MarkerColour!80!black}{\scriptsize\faLink} DOI: \href{http://dx.doi.org/10.1093/mnras/stae207}{10.1093/mnras/stae207})

\vspace*{0.3cm}

\circled{8}  Passegger et al, 2024 - 'The Compact Multi-Planet System {{GJ}} 9827 Revisited with {{ESPRESSO}}', Astronomy \& Astrophysics ( \textcolor{MarkerColour!80!black}{\scriptsize\faLink} DOI: \href{http://dx.doi.org/10.1051/0004-6361/202348592}{10.1051/0004-6361/202348592})

\vspace*{0.3cm}

\circled{7}  {Castro-Gonz{\'a}lez} et al, 2023 - 'An Unusually Low-Density Super-{{Earth}} Transiting the Bright Early-Type {{M-dwarf GJ}} 1018 ({{TOI-244}})', Astronomy \& Astrophysics ( \textcolor{MarkerColour!80!black}{\scriptsize\faLink} DOI: \href{http://dx.doi.org/10.1051/0004-6361/202346550}{10.1051/0004-6361/202346550})

\vspace*{0.3cm}

\circled{6}  {Balsalobre-Ruza} et al, 2023 - '{{KOBEsim}}: {{A Bayesian}} Observing Strategy Algorithm for Planet Detection in Radial Velocity Blind-Search Surveys', Astronomy \& Astrophysics ( \textcolor{MarkerColour!80!black}{\scriptsize\faLink} DOI: \href{http://dx.doi.org/10.1051/0004-6361/202243938}{10.1051/0004-6361/202243938})

\vspace*{0.3cm}

\circled{5}  Mascare{\~n}o et al, 2023 - 'Two Temperate {{Earth-mass}} Planets Orbiting the Nearby Star {{GJ1002}}', Astronomy \& Astrophysics ( \textcolor{MarkerColour!80!black}{\scriptsize\faLink} DOI: \href{http://dx.doi.org/10.1051/0004-6361/202244991}{10.1051/0004-6361/202244991})

\vspace*{0.3cm}

\circled{4}  Allart et al, 2022 - 'Automatic Model-Based Telluric Correction for the {{ESPRESSO}} Data Reduction Software: {{Model}} Description and Application to Radial Velocity Computation', Astronomy \& Astrophysics ( \textcolor{MarkerColour!80!black}{\scriptsize\faLink} DOI: \href{http://dx.doi.org/10.1051/0004-6361/202243629}{10.1051/0004-6361/202243629})

\vspace*{0.3cm}

\circled{3}  {Lillo-Box} et al, 2022 - 'The {{KOBE}} Experiment: {{K-dwarfs Orbited By}} Habitable {{Exoplanets}}: {{Project}} Goals, Target Selection, and Stellar Characterization', Astronomy \& Astrophysics ( \textcolor{MarkerColour!80!black}{\scriptsize\faLink} DOI: \href{http://dx.doi.org/10.1051/0004-6361/202243898}{10.1051/0004-6361/202243898})

\vspace*{0.3cm}

\circled{2}  Faria et al, 2022 - 'A Candidate Short-Period Sub-{{Earth}} Orbiting {{Proxima Centauri}}', Astronomy \& Astrophysics ( \textcolor{MarkerColour!80!black}{\scriptsize\faLink} DOI: \href{http://dx.doi.org/10.1051/0004-6361/202142337}{10.1051/0004-6361/202142337})

\vspace*{0.3cm}

\circled{1}  {Lillo-Box} et al, 2021 - '{{HD22496b}}: The First {{ESPRESSO}} Standalone Planet Discovery', Astronomy \& Astrophysics ( \textcolor{MarkerColour!80!black}{\scriptsize\faLink} DOI: \href{http://dx.doi.org/10.1051/0004-6361/202141714}{10.1051/0004-6361/202141714})

\vspace*{0.3cm}

\vspace*{0.2cm}

\mysubsection{Posters}

\vspace*{0.2cm}

\circled{7}  'The Paranal solar ESPRESSO Telescope - towards a resolved view of the Sun', Leiden, Exoplanets 5, 2024-06-16/2024-06-21

\vspace*{0.3cm}

\circled{6}  'A fully-Bayesian model for RV extraction', Leiden, Exoplanets 5, 2024-06-16/2024-06-21

\vspace*{0.3cm}

\circled{5}  'A Bayesian template matching approach applied to HARPS : towards the improvement of the RV precision', Online, European Astronomical Society Annual meeting 2021, 2021-06-28/2021-07-02

\vspace*{0.3cm}

\circled{4}  'A semi-Bayesian implementation of template matching for precise Radial Velocities', Online, Statistical challenges in Modern astronomy VII, 2021-07-07/2021-07-10

\vspace*{0.3cm}

\circled{3}  'A semi-Bayesian implementation of template matching for precise Radial Velocities', Online, Encontro Ciência 21, 2021-07-28/2021-07-30

\vspace*{0.3cm}

\circled{2}  'A Bayesian approach to precise Radial Velocities', Online, 30th Encontro Nacional de Astronomia e Astrofísica , 2020-09-09/2020-09-11

\vspace*{0.3cm}

\circled{1}  'ARCHI: pipeline for light curve extraction of CHEOPS background stars', Online, Europlanet Science Congress 2020, 2020-06-21/2020-07-09

\vspace*{0.3cm}

\vspace*{0.2cm}

\end{rubric}

\makerubric{Supervision}
\makerubric{Teaching}

\makerubric{outreach}

% \makerubric{skills}
\makerubric{misc}
 
% \makerubric{referee}
% %% Probably not the best way of doing it but what the heck, I just winged-it :p

\makerubrichead{References}

\begin{tabularx}{\textwidth}{@{}X X@{}}
\textbf{Prof X}\par
Professor\par
ABC University,\par 
Address.\par 
\makefield{\faEnvelopeO}{\url{abc@def.edu}}
& 
\textbf{Prof Y}\par
Professor\par
ABC University,\par 
Address.\par 
\makefield{\faEnvelopeO}{\url{abc@def.edu}}
\\
\end{tabularx}


\end{document}