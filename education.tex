%!TEX encoding = UTF8
%!TEX root =cv-llt.tex

\section{\langen{Education}\langpt{Educação}}


\langen{
% 2019-09--2024-07-09
\cventry{2019--2024}{\textbf{Doctor's Degree in Astronomy}}{Programa doutoral em Astronomia}{Faculdade de Ciências da Universidade do Porto}{}{Thesis title: \emph{A new paradigm for the estimation of precise stellar radial velocities.}}  
}
\langpt{
% 2019-09--2024-07-09
\cventry{2019--2024}{\textbf{Doutoramento em Astronomia}}{Programa doutoral em Astronomia}{Faculdade de Ciências da Universidade do Porto}{}{Tese: \emph{A new paradigm for the estimation of precise stellar radial velocities.}}  
}

\langen{
\cventry{2019--2024}{\textbf{M.Sc. Degree}}{Mestrado Integrado em Engenharia Física}{Faculdade de Ciências da Universidade do Porto}{}{Thesis title: \emph{An expansion to the CHEOPS mission official pipeline.}}  
}

\langt{
\cventry{2019--2024}{\textbf{Mestrado em Engenharia Física}}{Mestrado Integrado em Engenharia Física}{Faculdade de Ciências da Universidade do Porto}{}{Tese: \emph{An expansion to the CHEOPS mission official pipeline.}}  
}



% \newcommand{\MsCThesis}{
% \textbf{Msc Thesis: An expansion to the CHEOPS mission official pipeline (Jan - Sep 2019)\\}
% Development of an expansion to the official data reduction pipeline of the CHEOPS mission, allowing for the extraction of light curves from background stars in the images. The resulting light curves were also modelled using Gaussian Processes, with the goal of retrieving planetary parameters; \textit{Grade}: 19/20 
% }
